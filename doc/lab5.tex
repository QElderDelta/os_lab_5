\documentclass[a4paper, 12pt]{article}
\usepackage{cmap}
\usepackage[12pt]{extsizes}			
\usepackage{mathtext} 				
\usepackage[T2A]{fontenc}			
\usepackage[utf8]{inputenc}			
\usepackage[english,russian]{babel}
\usepackage{setspace}
\singlespacing
\usepackage{amsmath,amsfonts,amssymb,amsthm,mathtools}
\usepackage{fancyhdr}
\usepackage{soulutf8}
\usepackage{euscript}
\usepackage{mathrsfs}
\usepackage{listings}
\pagestyle{fancy}
\usepackage{indentfirst}
\usepackage[top=10mm]{geometry}
\rhead{}
\lhead{}
\renewcommand{\headrulewidth}{0mm}
\usepackage{tocloft}
\renewcommand{\cftsecleader}{\cftdotfill{\cftdotsep}}
\usepackage[dvipsnames]{xcolor}

\lstdefinestyle{mystyle}{ 
	keywordstyle=\color{OliveGreen},
	numberstyle=\tiny\color{Gray},
	stringstyle=\color{BurntOrange},
	basicstyle=\ttfamily\footnotesize,
	breakatwhitespace=false,         
	breaklines=true,                 
	captionpos=b,                    
	keepspaces=true,                 
	numbers=left,                    
	numbersep=5pt,                  
	showspaces=false,                
	showstringspaces=false,
	showtabs=false,                  
	tabsize=2
}

\lstset{style=mystyle}

\begin{document}
\thispagestyle{empty}	
\begin{center}
	Московский авиационный институт
	
	(Национальный исследовательский университет)
	
	Факультет "Информационные технологии и прикладная математика"
	
	Кафедра "Вычислительная математика и программирование"
	
\end{center}
\vspace{40ex}
\begin{center}
	\textbf{\large{Лабораторная работа №5 по курсу\linebreak \textquotedblleft Операционные системы\textquotedblright}}
\end{center}
\vspace{35ex}
\begin{flushright}
	\textit{Студент: } Живалев Е.А.
	
	\vspace{2ex}
	\textit{Группа: } М8О-206Б
	
	\vspace{2ex}
	\textit{Преподаватель: } Соколов А.А.
	
	\vspace{2ex}
	\textit{Вариант: } 7
	
	\vspace{2ex}
	\textit{Оценка: } \underline{\quad\quad\quad\quad\quad\quad}
	
	 \vspace{2ex}
	\textit{Дата: } \underline{\quad\quad\quad\quad\quad\quad}
	
	\vspace{2ex}
	\textit{Подпись: } \underline{\quad\quad\quad\quad\quad\quad}
	
\end{flushright}

\vspace{5ex}

\begin{vfill}
	\begin{center}
		Москва, 2019
	\end{center}	
\end{vfill}
\newpage

\section{Задание}

Требуется создать динамическую библиотеку, которая реализует определенный Функционал - работу с массивом, содержащим целые 32-битные числа.

\section{Описание работы программы}

В файле arrayAPI.c реализованы следующие функции для работы с массивом: arrayCreate, arrayInsert, arrayGet, arrayDelete, arrayResize, arrayDestroy, arrayPrint. В файле main\_dynamic.c эти функции загружаются в память, выделенную для программы  
\newpage

\section{Исходный код}
\textbf{\large{main\_dynamic.c}}
\lstinputlisting[language=C]{../src/main_dynamic.c}

\textbf{\large{main\_static.c}}
\lstinputlisting[language=C]{../src/main_static.c}

\textbf{\large{arrayAPI.h}}
\lstinputlisting[language=C]{../src/arrayAPI.h}

\textbf{\large{arrayAPI.h}}
\lstinputlisting[language=C]{../src/arrayAPI.c}
\newpage
\section{Консоль}

\begin{verbatim}
qelderdelta@qelderdelta-UX331UA:~/Study/OS/os_lab_5/src$ ./dynamic
1 - create array with given size
2 - insert element to array at given position
3 - get element value on given position
4 - delete element from given position
5 - resize array to given size
6 - print array
0 - exit
1
Enter value
3
2
Enter position
0 1
2   
Enter position
1 2
2  
Enter position
2 3
6
1 2 3 
3
Enter position
0
1
0
\end{verbatim}
\newpage
\section{Выводы}

В ходе выполнения лабораторной работы я познакомился с такой полезной вещью, как использование динамических библиотек, которые позволяют ускорить работу со сторонними библиотеками, поскольку позволяют включать в программу только необходимые функции, а не все, реализованные в библиотеке.
\end{document}